% Options for packages loaded elsewhere
\PassOptionsToPackage{unicode}{hyperref}
\PassOptionsToPackage{hyphens}{url}
\PassOptionsToPackage{dvipsnames,svgnames,x11names}{xcolor}
%
\documentclass[
]{article}
\title{Predicting readmission probability for diabetes inpatients}
\author{Modern Data Mining}
\date{}

\usepackage{amsmath,amssymb}
\usepackage{lmodern}
\usepackage{iftex}
\ifPDFTeX
  \usepackage[T1]{fontenc}
  \usepackage[utf8]{inputenc}
  \usepackage{textcomp} % provide euro and other symbols
\else % if luatex or xetex
  \usepackage{unicode-math}
  \defaultfontfeatures{Scale=MatchLowercase}
  \defaultfontfeatures[\rmfamily]{Ligatures=TeX,Scale=1}
\fi
% Use upquote if available, for straight quotes in verbatim environments
\IfFileExists{upquote.sty}{\usepackage{upquote}}{}
\IfFileExists{microtype.sty}{% use microtype if available
  \usepackage[]{microtype}
  \UseMicrotypeSet[protrusion]{basicmath} % disable protrusion for tt fonts
}{}
\makeatletter
\@ifundefined{KOMAClassName}{% if non-KOMA class
  \IfFileExists{parskip.sty}{%
    \usepackage{parskip}
  }{% else
    \setlength{\parindent}{0pt}
    \setlength{\parskip}{6pt plus 2pt minus 1pt}}
}{% if KOMA class
  \KOMAoptions{parskip=half}}
\makeatother
\usepackage{xcolor}
\IfFileExists{xurl.sty}{\usepackage{xurl}}{} % add URL line breaks if available
\IfFileExists{bookmark.sty}{\usepackage{bookmark}}{\usepackage{hyperref}}
\hypersetup{
  pdftitle={Predicting readmission probability for diabetes inpatients},
  pdfauthor={Modern Data Mining},
  colorlinks=true,
  linkcolor={Maroon},
  filecolor={Maroon},
  citecolor={Blue},
  urlcolor={blue},
  pdfcreator={LaTeX via pandoc}}
\urlstyle{same} % disable monospaced font for URLs
\usepackage[margin=1in]{geometry}
\usepackage{graphicx}
\makeatletter
\def\maxwidth{\ifdim\Gin@nat@width>\linewidth\linewidth\else\Gin@nat@width\fi}
\def\maxheight{\ifdim\Gin@nat@height>\textheight\textheight\else\Gin@nat@height\fi}
\makeatother
% Scale images if necessary, so that they will not overflow the page
% margins by default, and it is still possible to overwrite the defaults
% using explicit options in \includegraphics[width, height, ...]{}
\setkeys{Gin}{width=\maxwidth,height=\maxheight,keepaspectratio}
% Set default figure placement to htbp
\makeatletter
\def\fps@figure{htbp}
\makeatother
\setlength{\emergencystretch}{3em} % prevent overfull lines
\providecommand{\tightlist}{%
  \setlength{\itemsep}{0pt}\setlength{\parskip}{0pt}}
\setcounter{secnumdepth}{-\maxdimen} % remove section numbering
\ifLuaTeX
  \usepackage{selnolig}  % disable illegal ligatures
\fi

\begin{document}
\maketitle

{
\hypersetup{linkcolor=}
\setcounter{tocdepth}{4}
\tableofcontents
}
\hypertarget{instructions}{%
\section{Instructions}\label{instructions}}

\begin{itemize}
\tightlist
\item
  This is a project. Well organized and well presented write-up is one
  major motivation here. Please see the section on \texttt{Write\ up}
  for details.
\item
  There is no single correct answer.\\
\item
  The entire write up should not be more than \textbf{5} pages. All the
  R-codes should be hidden. Any R-output used should be formatted
  neatly. You may put all supporting documents, graphics, or other
  exhibits into an Appendix, which is not counted in the 5 page limit.
\end{itemize}

\hypertarget{introduction}{%
\section{Introduction}\label{introduction}}

\hypertarget{background}{%
\subsection{Background}\label{background}}

Diabetes is a chronic medical condition affecting millions of Americans,
but if managed well, with good diet, exercise and medication, patients
can lead relatively normal lives. However, if improperly managed,
diabetes can lead to patients being continuously admitted and readmitted
to hospitals. Readmissions are especially serious - they represent a
failure of the health system to provide adequate support to the patient
and are extremely costly to the system. As a result, the Centers for
Medicare and Medicaid Services announced in 2012 that they would no
longer reimburse hospitals for services rendered if a patient was
readmitted with complications within 30 days of discharge.

Given these policy changes, being able to identify and predict those
patients most at risk for costly readmissions has become a pressing
priority for hospital administrators.

\hypertarget{goal-of-the-study}{%
\subsection{Goal of the study}\label{goal-of-the-study}}

In this project, we shall explore how to use the techniques we have
learned in order to help better manage diabetes patients who have been
admitted to a hospital. Our goal is to avoid patients being readmitted
within 30 days of discharge, which reduces costs for the hospital and
improves outcomes for patients. If we could identify important factors
relating to the chance of a patient being readmitted within 30 days of
discharge, effective intervention could be done to reduce the chance of
being readmitted. Also if we could predict one's chance being readmitted
well, actions can be taken.

\hypertarget{the-data}{%
\subsection{The data}\label{the-data}}

The original data is from the
\href{https://archive.ics.uci.edu/ml/datasets/Diabetes+130-US+hospitals+for+years+1999-2008}{Center
for Clinical and Translational Research} at Virginia Commonwealth
University. It covers data on diabetes patients across 130 U.S.
hospitals from 1999 to 2008. There are over 100,000 unique hospital
admissions in this dataset, from \textasciitilde70,000 unique patients.
The data includes demographic elements, such as age, gender, and race,
as well as clinical attributes such as tests conducted,
emergency/inpatient visits, etc. Refer to the original documentation for
more details on the dataset. Three former students Spencer Luster,
Matthew Lesser and Mridul Ganesh, brought this data set into the class
and did a wonderful final project. We will use a subset processed by the
group but with a somewhat different objective.

Data needed (see detailed information below):

\begin{itemize}
\tightlist
\item
  \textbf{\texttt{diabetic.data.csv}}
\item
  \textbf{\texttt{readmission.csv}}
\end{itemize}

\hypertarget{characteristics-of-the-data-set}{%
\subsubsection{Characteristics of the Data
Set}\label{characteristics-of-the-data-set}}

All observations have five things in common:

\begin{enumerate}
\def\labelenumi{\arabic{enumi}.}
\tightlist
\item
  They are all hospital admissions
\item
  Each patient had some form of diabetes
\item
  The patient stayed for between 1 and 14 days.
\item
  The patient had laboratory tests performed on him/her.
\item
  The patient was given some form of medication during the visit.
\end{enumerate}

The data was collected during a ten-year period from 1999 to 2008. There
are over 100,000 unique hospital admissions in the data set, with
\textasciitilde70,000 unique patients.

\hypertarget{description-of-variables}{%
\subsubsection{Description of
variables}\label{description-of-variables}}

The dataset used covers \textasciitilde50 different variables to
describe every hospital diabetes admission. In this section we give an
overview and brief description of the variables in this dataset.

\textbf{1) Patient identifiers:}

\begin{enumerate}
\def\labelenumi{\alph{enumi}.}
\tightlist
\item
  \texttt{encounter\_id}: unique identifier for each admission
\item
  \texttt{patient\_nbr}: unique identifier for each patient
\end{enumerate}

\textbf{2) Patient Demographics:}

\texttt{race}, \texttt{age}, \texttt{gender}, \texttt{weight} cover the
basic demographic information associated with each patient.
\texttt{Payer\_code} is an additional variable that identifies which
health insurance (Medicare /Medicaid / Commercial) the patient holds.

\textbf{3) Admission and discharge details:}

\begin{enumerate}
\def\labelenumi{\alph{enumi}.}
\tightlist
\item
  \texttt{admission\_source\_id} and \texttt{admission\_type\_id}
  identify who referred the patient to the hospital (e.g.~physician
  vs.~emergency dept.) and what type of admission this was (Emergency
  vs.~Elective vs.~Urgent).
\item
  \texttt{discharge\_disposition\_id} indicates where the patient was
  discharged to after treatment.
\end{enumerate}

\textbf{4) Patient Medical History:}

\begin{enumerate}
\def\labelenumi{\alph{enumi}.}
\tightlist
\item
  \texttt{num\_outpatient}: number of outpatient visits by the patient
  in the year prior to the current encounter
\item
  \texttt{num\_inpatient}: number of inpatient visits by the patient in
  the year prior to the current encounter
\item
  \texttt{num\_emergency}: number of emergency visits by the patient in
  the year prior to the current encounter
\end{enumerate}

\textbf{5) Patient admission details:}

\begin{enumerate}
\def\labelenumi{\alph{enumi}.}
\tightlist
\item
  \texttt{medical\_specialty}: the specialty of the physician admitting
  the patient
\item
  \texttt{diag\_1}, \texttt{diag\_2}, \texttt{diag\_3}: ICD9 codes for
  the primary, secondary and tertiary diagnoses of the patient. ICD9 are
  the universal codes that all physicians use to record diagnoses. There
  are various easy to use tools to lock up what individual codes mean
  (Wikipedia is pretty decent on its own)
\item
  \texttt{time\_in\_hospital}: the patient's length of stay in the
  hospital (in days)
\item
  \texttt{number\_diagnoses}: Total no. of diagnosis entered for the
  patient
\item
  \texttt{num\_lab\_procedures}: No.~of lab procedures performed in the
  current encounter
\item
  \texttt{num\_procedures}: No.~of non-lab procedures performed in the
  current encounter
\item
  \texttt{num\_medications}: No.~of distinct medications prescribed in
  the current encounter
\end{enumerate}

\textbf{6) Clinical Results:}

\begin{enumerate}
\def\labelenumi{\alph{enumi}.}
\tightlist
\item
  \texttt{max\_glu\_serum}: indicates results of the glucose serum test
\item
  \texttt{A1Cresult}: indicates results of the A1c test
\end{enumerate}

\textbf{7) Medication Details:}

\begin{enumerate}
\def\labelenumi{\alph{enumi}.}
\tightlist
\item
  \texttt{diabetesMed}: indicates if any diabetes medication was
  prescribed
\item
  \texttt{change}: indicates if there was a change in diabetes
  medication
\item
  \texttt{24\ medication\ variables}: indicate whether the dosage of the
  medicines was changed in any manner during the encounter
\end{enumerate}

\textbf{8) Readmission indicator:}

Indicates whether a patient was readmitted after a particular admission.
There are 3 levels for this variable: ``NO'' = no readmission,
``\textless{} 30'' = readmission within 30 days and ``\textgreater{}
30'' = readmission after more than 30 days. The 30 day distinction is of
practical importance to hospitals because federal regulations penalize
hospitals for an excessive proportion of such readmissions.

To save your time we are going to use some data sets cleaned by the
group. Thus, we provide two datasets:

\textbf{\texttt{diabetic.data.csv}} is the original data. You may use it
for the purpose of summary if you wish. You will see that the original
data can't be used directly for your analysis, yet.

\textbf{\texttt{readmission.csv}} is a cleaned version and they are
modified in the following ways:

\begin{enumerate}
\def\labelenumi{\arabic{enumi})}
\item
  \texttt{Payer\ code}, \texttt{weight} and \texttt{Medical\ Specialty}
  are not included since they have a large number of missing values.
\item
  Variables such as \texttt{acetohexamide},
  \texttt{glimepiride.pioglitazone}, \texttt{metformin.rosiglitazone},
  \texttt{metformin.pioglitazone} have little variability, and are as
  such excluded. This also includes the following variables:
  \texttt{chlorpropamide}, \texttt{acetohexamide}, \texttt{tolbutamide},
  \texttt{acarbose}, \texttt{miglitor}, \texttt{troglitazone},
  \texttt{tolazamide}, \texttt{examide}, \texttt{citoglipton},
  \texttt{glyburide.metformin}, \texttt{glipizide.metformin}, and
  \texttt{glimepiride.pioglitazone}.
\item
  Some categorical variables have been regrouped. For example,
  \texttt{Diag1\_mod} keeps some original levels with large number of
  patients and aggregates other patients as \texttt{others}. This
  process is known as `binning.'
\item
  The event of interest is \textbf{readmitted within \textless{} 30
  days}. Note that you need to create this response first by regrouping
  \textbf{Readmission indicator}!
\end{enumerate}

\hypertarget{research-approach}{%
\section{Research approach}\label{research-approach}}

From the \emph{Goals} section above, your study should respond to the
following:

\hypertarget{analyses-suggested}{%
\subsection{Analyses suggested}\label{analyses-suggested}}

\begin{enumerate}
\def\labelenumi{\arabic{enumi})}
\tightlist
\item
  Identify important factors that capture the chance of a readmission
  within 30 days.
\end{enumerate}

The set of available predictors is not limited to the raw variables in
the data set. You may engineer any factors using the data, that you
think will improve your model's quality.

\begin{enumerate}
\def\labelenumi{\arabic{enumi})}
\setcounter{enumi}{1}
\item
  For the purpose of classification, propose a model that can be used to
  predict whether a patient will be a readmit within 30 days. Justify
  your choice. Hint: use a decision criterion, such as AUC, to choose
  among a few candidate models.
\item
  Based on a quick and somewhat arbitrary guess, we estimate \textbf{it
  costs twice as much} to mislabel a readmission than it does to
  mislabel a non-readmission. Based on this risk ratio, propose a
  specific classification rule to minimize the cost. If you find any
  information that could provide a better cost estimate, please justify
  it in your write-up and use the better estimate in your answer.
\end{enumerate}

Suggestion: You may use any of the methods covered so far in parts 1)
and 2), and they need not be the same. Also keep in mind that a
training/testing data split may be necessary.

\begin{enumerate}
\def\labelenumi{\arabic{enumi})}
\setcounter{enumi}{3}
\tightlist
\item
  We suggest you to split the data first to Training/Testing/Validation
  data:
\end{enumerate}

\begin{itemize}
\item
  Use training/testing data to land a final model (If you only use LASSO
  to land a final model, we will not need testing data since all the
  decisions are made with cross-validations.)
\item
  Evaluate the final model with the validation data to give an honest
  assessment of your final model.
\end{itemize}

\hypertarget{the-write-up}{%
\section{The write up}\label{the-write-up}}

As you all know, it is very important to present your findings well. To
achieve the best possible results you need to understand your audience.

Your target audience is a manager within the hospital organization. They
hold an MBA, are familiar with medical terminology (though you do not
need any previous medical knowledge), and have gone through a similar
course to our Modern Data Mining with someone like your professor. You
can assume thus some level of technical familiarity, but should not let
the paper be bogged down with code or other difficult to understand
output.

Note then that the most important elements of your report are the
clarity of your analysis and the quality of your proposals.

A suggested outline of the report would include the following
components:

\begin{enumerate}
\def\labelenumi{\arabic{enumi})}
\tightlist
\item
  Executive Summary
\end{enumerate}

\begin{itemize}
\tightlist
\item
  This section should be accessible by people with very little
  statistical background (avoid using technical words and no direct R
  output is allowed)
\item
  Give a background of the study. You may check the original website or
  other sources to fill in some details, such as to why the questions we
  address here are important.
\item
  A quick summary about the data.
\item
  Methods used and the main findings.
\item
  You may use clearly labelled and explained visualizations.
\item
  Issues, concerns, limitations of the conclusions. This is an
  especially important section to be honest in - we might be Penn
  students, but we are statisticians today.
\end{itemize}

\begin{enumerate}
\def\labelenumi{\arabic{enumi})}
\setcounter{enumi}{1}
\tightlist
\item
  Detailed process of the analysis
\end{enumerate}

\begin{enumerate}
\def\labelenumi{\roman{enumi})}
\tightlist
\item
  Data Summary /EDA
\end{enumerate}

\begin{itemize}
\tightlist
\item
  Nature of the data, origin
\item
  Necessary quantitative and graphical summaries
\item
  Are there any problems with the data?
\item
  Which variables are considered as input
\end{itemize}

\begin{enumerate}
\def\labelenumi{\roman{enumi})}
\setcounter{enumi}{1}
\tightlist
\item
  Analyses
\end{enumerate}

\begin{itemize}
\tightlist
\item
  Various appropriate statistical methods: e.g.~glmnet
\item
  Comparisons various models
\item
  Final model(s)
\end{itemize}

\begin{enumerate}
\def\labelenumi{\roman{enumi})}
\setcounter{enumi}{2}
\tightlist
\item
  Conclusion
\end{enumerate}

\begin{itemize}
\tightlist
\item
  Summarize results and the final model
\item
  Final recommendations
\end{itemize}

Maintain a good descriptive flow in the text of your report. Use
Appendices to display lengthy output.

\begin{enumerate}
\def\labelenumi{\roman{enumi})}
\setcounter{enumi}{2}
\tightlist
\item
  Appendix
\end{enumerate}

\begin{itemize}
\tightlist
\item
  Any thing necessary to keep but for which you don't want them to be in
  the main report.
\end{itemize}

\end{document}
